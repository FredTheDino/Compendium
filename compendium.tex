\documentclass{article}
\usepackage[utf8]{inputenc}
\usepackage{amsfonts}
\usepackage{amsmath}
\usepackage{amssymb}
\setlength{\parindent}{0pt}
\setlength{\parskip}{\baselineskip}

\title{Rekursion och Induktion}
\author{Edvard Thörnros, Erik Mattfolk}

\usepackage[a4paper, includeheadfoot, left=2cm, right=2cm, top=2cm]{geometry}

\begin{document}
	\maketitle
	\newpage

	\section{Mängder}
	\subsection{Grundläggande definitioner}
	En mängd är en sammling element, den behöver inte vara hetrogen eller sorterad. Men varje element kan endast förekomma en gång.
	$$
		A = \{ 1, 2, 3 \}
		\quad
		B = \{ a, b, c, d\}
	$$
	$$
		C = \{x : x \in \mathbb{Z}_{+} \} 
		\quad
		D = \{ \{ 1 \}, 1\}
	$$

	$ A, B $ är mängder, $A$ är delmängd till $B$ om ALLA element i $A$ finns i $B$.
	$$ A \subseteq B $$

	$ A, B $ är mängder, $A$ är äkta delmängd till $B$ om ALLA element i $A$ finns i $B$. Och $A \not= B$
	$$ A \subset B $$

	$ A, B $ är mängder, $A$ är delmängd till $B$ om ALLA element i $A$ finns i $B$.
	$$ A \subseteq B $$

	$ (a, b) $ är ett ordnat par.

	$$ (a, b) = (c, d) \to a = c, b = d $$
	$$ (a, b) = (b, a) \to a \not= b $$

	$A$ är en mängd. Kardinaliteten $|A|$ är antalet element som finns i $A$ om $A$ är en ändlig mängd, annars är den oändlig.

	Om $A \cap B = \emptyset $ betyder det att $A$ och $B$ är disjunkta.
	
	\begin{center}
		$P(A) = \{S, S \in A\}$
	\end{center}

	\subsection{Operationer}
	Union: $ A \cup B = \{x : x \in A \text{ eller } x \in B\} $\\
	Snitt: $ A \cap B = \{x : x \in B \text{ och } x \in A\} $\\
	Symetrisk Differans: $ A \bigtriangleup B = \{x : x \in A \text{ eller } x \in B, x \not\in A \cap B \} $\\
	Differans: $ A \setminus B = \{x : x \in A \text{ och } x \not\in B \} $\\
	Komplement: låt $ \upsilon $ vara grund mängden, där $ A \subset \upsilon $. Låt då $ \overline{A} = \upsilon \setminus A $\\
	Karteisika Produkt: $A \times B = \{ (a, b) : a \in A, b \in B\} $\\
	Potensmängden: $P(A)$ är alla delmängder till $A$. \\

	\subsection{Användbara Hälger}
	Naturligatal $\mathbb{N} = \{0, 1, 2, 3, 4...\}$\\
	Heltal $\mathbb{Z} = \{0, 1, -1, 2, -2, 3, -3...\}$\\
	Posetiva heltal $\mathbb{Z}_{+} = \{1, 2, 3, 4, 5...\}$\\
	Negative heltal $\mathbb{Z}_{-} = \{-1, -2, -3, -4, -5...\}$\\
	Rationella $\mathbb{Q} = \{\frac{p}{q} : p \in \mathbb{Z}, q \in \mathbb{Z}_{+}\}$\\

	Irattionella tal är alla tal som inte kan skrivas som ett bråk men som finns på den reela tallinjen. Här noteras settet som $\mathbb{P}$

	Reela tal \( \mathbb{R} = \mathbb{Q} \cup \mathbb{P} \)

	Tomma mängden $\emptyset = \{\}$

	Låt $A = \{a, b\}, B = \{b, c\}, \upsilon = \{a, b, c, d\}$ då följer:

	$$ A \cup B = \{a, b, c\} $$
	$$ A \cap B = \{b\} $$
	$$ A \bigtriangleup B = \{a, c\} $$
	$$ A \setminus B = \{a\} $$
	$$ \overline{A} = \{c, d\} $$
	$$ |A| = 2 $$
	$$ A \times B = \{(a, b), (a, c), (b, b), (b, c)\}$$

	Låt $A = \{a, b, c\}$ då följer: 
	$$ P(A) = \{\emptyset, \{a\}, \{b\}, \{c\}, \{a, b\}, \{a, c\}, \{b, c\}, \{a, b, c\}\} $$

	\subsection{Sats 1, Kardinaliteten av den kartesiska produkten}
	Låt $A \not= \emptyset, B \not= \emptyset$. Det medföjer då att:
	$$ |A \times B| = |A| * |B| $$ 
	Detta kan bevisas genom ett triviellt logiskt argument, nämligen att vi kombinerar alla element i $A$ med alla element i $B$. 
	Detta leder till att vi för varje $a \in A$ får $|B|$ möjliga kombinationer. Och vi har $|A|$ st $a$, vilket ger oss $ |A| * |B| $ möjliga kombinationer. 
	
	\subsection{Sats 2, Kardinalitet efter union med två mägnder}
	Låt $A, B$ vara två icke tomma mängder.
	$$ |A \cup B| = |A| + |B| - |A \cap B| $$
	Bevis för denna sats är triviell.
	
	\subsection{Sats 3, Kardinalitet efter union med tre mängder}
	Låt $A, B, C$ vara tre icke tomma mängder.
	$$ |A \cup B \cup C| = |A| + |B| + |C| - |A \cap B| - |A \cap C| + |A \cap B \cap C| $$
	Bevis för denna sats är också triviell, då man lätt kan övertyga sig själv med hjälp av ett venn diagram av de mängder man tar bort och lägger till. 

	\subsection{Sats 4, Distrubutiva regeln}
	Låt $A, B, C$ vara tre mängder, då gäller: 
	$$ A \cap (B \cup C) = (A \cap B) \cup (A \cap C) $$
	$$ A \cup (B \cap C) = (A \cup B) \cap (A \cup C) $$
	\newpage
		
	\section{Induktion}
	\subsection{Induktionsprincipen}
	Påstående \\ 
	Låt $P(n)$ vara ett påstående \\
	\underline{Bassteg:} \\
	Börja med att bevisa att påståendet är sant för $n = n_{0}$ \\
	\underline{Antagande}: \\
	Antag att påståendet gäller för något $k \geq n$ dvs. $P(k)$ \\
	\underline{Induktionssteg}: \\
	Visa med hjälp av antagandet att påståendet stämmer då $k = p + 1$ \\
	\underline{Slutsats:} \\
	Påståendet stämmer enligt Induktionsprincipen \\
	\subsection{Alternativa Induktionsprincipen}
	Påstående \\
	Antag att $P(n)$ beskriver en rekursiv talföljd $a_{n}$ med 2 eller mer steg \\
	\underline{Bassteg:} \\
	Visa att påståendet stämmer för $n_{0}, ... , n_{1}$ \\
	\underline{Antagande:} \\
	Antag att påståendet stämmer för något $k \geq n_{1}$ Speciellt att $P(k-n_{1}), ... , P(k)$ stämmer \\
	\underline{Induktionssteg:} \\
	Visa med hjälp av antagandet att påståendet stämmer då $k = p + 1$ \\
	\underline{Slutsats:} \\
	Påståendet stämmer enligt den Alternativa Induktionsprincipen \\
	\section{Kombinatorik}
	\subsection{Definitioner}
	\subsubsection{Fakultet}
	$$
	\begin{cases}
		n! = n * (n - 1)! & \text{for }n > 0\\    
		n! = 1 & \text{for } n = 0\\
	\end{cases}
	$$
	\subsubsection{Additionsprinciper}
	Låt $A_1, A_2, A_3, \dots ,A_k$ vara parvis disjunkta alltså, $A_i \cap A_j = \emptyset$ då gäller att:
	$$
		|A_1 \cup A_2 \dots \cup A_k| = |A_1| + |A_2| + \dots |A_k|
	$$
	\subsubsection{Multiplicationsprincipen}


	\section{Talteori}

	\section{Relationer}

	\section{Modulär Aritmatik}
	
	\subsection{Kongruent Modulo}
	$a$ är \textbf{kongruent modulo} $n$ med $b$ om $n | (a-b)$ annars \textbf{Inkongruent} \\
	Detta betecknas	$a \equiv b \mod n$

	\subsection{Räkneregler för kongruensräkninggng}
	Låt $n \geq 2$ vara ett heltal. Antag att $a \equiv b \mod n$ och $c \equiv d \mod n$ \\
	Då gäller

	\section{Grafer}

\end{document}
