\documentclass{article}
\usepackage[utf8]{inputenc}
\usepackage{amsfonts}
\usepackage{amsmath}
\usepackage{amssymb}
\setlength{\parindent}{0pt}
\setlength{\parskip}{\baselineskip}

\title{Rekursion och Induktion}
\author{Edvard Thörnros, Erik Mattfolk}

\begin{document}
	\maketitle
	\newpage

	\section{Mängder}

	\section{Induktion}
	\subsection{Induktionsprincipen}
	Påstående \\ 
	Låt $P(n)$ vara ett påstående \\
	\underline{Bassteg:} \\
	Börja med att bevisa att påståendet är sant för $n = n_{0}$ \\
	\underline{Antagande}: \\
	Antag att påståendet gäller för något $k \geq n$ dvs. $P(k)$ \\
	\underline{Induktionssteg}: \\
	Visa med hjälp av antagandet att påståendet stämmer då $k = p + 1$ \\
	\underline{Slutsats:} \\
	Påståendet stämmer enligt Induktionsprincipen \\
	\subsection{Alternativa Induktionsprincipen}
	Påstående \\
	Antag att $P(n)$ beskriver en rekursiv talföljd $a_{n}$ med 2 eller mer steg \\
	\underline{Bassteg:} \\
	Visa att påståendet stämmer för $n_{0}, ... , n_{1}$ \\
	\underline{Antagande:} \\
	Antag att påståendet stämmer för något $k \geq n_{1}$ Speciellt att $P(k-n_{1}), ... , P(k)$ stämmer \\
	\underline{Induktionssteg:} \\
	Visa med hjälp av antagandet att påståendet stämmer då $k = p + 1$ \\
	\underline{Slutsats:} \\
	Påståendet stämmer enligt den Alternativa Induktionsprincipen \\
	\section{Kombinatorik}

	\section{Talteori}

	\section{Relationer}

	\section{Modulär Aritmatik}

	\section{Grafer}

\end{document}
